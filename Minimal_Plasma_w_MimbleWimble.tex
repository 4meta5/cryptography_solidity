
\documentclass{amsart}
\usepackage[english]{babel}
\usepackage[utf8]{inputenc}
\usepackage{amssymb}
\usepackage{mathtools}
\usepackage{amsmath}
\newtheorem{thm}{Theorem}[section]
\newtheorem{prop}[thm]{Proposition}
\newtheorem{lem}[thm]{Lemma}
\newtheorem{cor}[thm]{Corollary}
\theoremstyle{definition}
\newtheorem{definition}[thm]{Definition}
\newtheorem{example}[thm]{Example}
\newtheorem{dis}[thm]{Discussion}
\newtheorem{rem}[thm]{remark}
\newcounter{casecount}
\setcounter{casecount}{0}
\newenvironment{case}{\refstepcounter{casecount}\textbf{Case \arabic{casecount}:}}{}

\usepackage{etoolbox}
\AtBeginEnvironment{proof}{\setcounter{casecount}{0}}
\newtheorem{remark}[thm]{Remark}
\numberwithin{equation}{section}
\newcommand{\R}{\mathbb{R}}  % The real numbers.
\newcommand{\Q}{\mathbb{Q}}  % The rational numbers.
\newcommand{\C}{\mathbb{C}}  % The rational numbers.
\newcommand{\Z}{\mathbb{Z}}
\newcommand{\D}{\mathcal{O}_K}
\newcommand{\p}{\mathfrak{p}}
\newcommand{\B}{\mathfrak{B}}
\DeclareMathOperator{\dist}{dist} % The distance.

\usepackage[backend=biber]{biblatex}
\addbibresource{references.bib}
\author{Amar Singh}
\begin{document}
\title{Minimal Plasma Implementation with MimbleWimble Protocol}
\maketitle
\begin{abstract}
Providing scalability by encoding a blockchain in smart contracts within the Ethereum blockchain. This will be implemented with a block leader who submits a hash to the main Ethereum chain every x blocks, and then the participants in the Plasma chain can challenge the hash if they have verified the work and discovered a mistake.\\

The choice to do this implementation with the MimbleWimble protocol follows from the fact that MimbleWimble minimizes the computation done on the blockchain while also providing important privacy guarantees.\\

In the process of building this, I will learn a lot about smart contract vulnerabilities, specificities of working with the Ethereum Virtual Machine, and how to code in Solidity. Moreover, I will discover the important open problems with respect to Plasma and attain a better understanding of the areas where there can be improvement.\\

Extensions: incorporating zk-SNARKs that are somehow recursive and/or incorporate Bulletproofs by Benedict Buenz in order to easily verify and prove the MimbleWimble ECC process--this would increase the scalability of this chain significantly.\\
\end{abstract}
\tableofcontents

%could put this in the abstract of have an intro section
\section{Higher Level Overview}
-say what Plasma is first \cite{1}\\
-explain how Plasma can be implemented with any consensus mechanism\\
-we chose MimbleWimble because it uses elliptic curve cryptography in a clever way to minimize data throughput\\
-using bulletproofs for the range proofs\\
-how will this work and what are the main problems (block withholding and chain exits)\\
-later, will discuss how we outsource availability to Ethereum\\
-increasing safety by using a token for all transactions on the plasma chain (incorporate Zaki Manian's staking token model argument and logic...it could also allow for eventual interoperability with cross-chain atomic swaps and other new technology)\\
-investigating the process of encoding another blockchain within this blockchain (and having a new token and then discussing what happens when you add layers and how the token value should decrease at some rate presumably according to losses in liquidity at each level)\\

\section{Introduction to Plasma}
-explain how this delayed consistency thing would work and pull language and the implementation idea from what I currently understand and notes on the paper

\subsection{Outsourcing Availability to the Ethereum Blockchain}
[LATER IN THE PAPER, I NEED TO DISCUSS THE WAY BY WHICH WE OUTSOURCE AVAILABILITY TO THE ETHEREUM BLOCKCHAIN AND WHY THIS IS IMPORTANT BECAUSE IT MEANS THE PROTOCOL ISNT NECESSARILY ROBUST AGAINST ATTACKS ON THE ROOT BLOCKCHAIN]


\section{MimbleWimble}
The Grin project is working on implementing the MimbleWimble protocol in Rust. The main goal and characteristics of the Grin project are [CITATION]:\\
1) Privacy by default. This enables complete fungibility without precluding the ability to selectively disclose information as needed.\\
2) Scales with the number of users and not the number of transactions, with very large space saving compared to other blockchains.\\
3) Strong and proven cryptography. MimbleWimble relies on Elliptic Curve Cryptography which has been tried and tested for decades. \\
4) Design simplicity that makes it easy to audit and maintain over time. \\

//REWORD THIS PARAGRAPH BUT BASICALLY #5 FOR GRIN IS DECENTRALIZED MINING AND OURS WILL BE CENTRALIZED//\\
The Grin project also incorporates the Cuckoo Cycle hashing algorithm [CITATION; JOHN TROMP] to encourage mining decentralization. However, instead of providing a decentralized consensus mechanism, we will follow in the spirit of Lightning [CITATION] with an implementation of the Plasma protocol [CITATION] in order to provide delayed guarantees of consistency via allowing the reversion of invalid state transitions with a submission protocols for proofs related to the blockchain's computation. These proofs can be submitted by participants in the state channel if they notice that the block dictator has submitted something that is incorrect to the Ethereum blockchain...MOST OF THIS PARAGRAPH WILL BE COVERED MORE IN DEPTH IN THE FIRST SECTION ON PLASMA AND HOW WE'RE GOING TO USE IT.\\

COMMUNICATE THE FACT THAT THIS REALLY RELIES ON USING ELLIPTIC CURVE CRYPTOGRAPHY AND IS STRONG BECAUSE OF THIS. REALLY, IT ALL RELIES ON THE DISCRETE LOGARITHM PROBLEM BEING ESPECIALLY QUITE HARD AND THIS IS A WELL TESTED PROBLEM IN CRYPTOGRAPHY. MOREOVER, EXPLAIN HOW THE FIRST SECTION WON'T BE SUPER INTENSE BUT INSTEAD WILL BE MAINLY USED TO UNDERSTAND NOTATION.\\

\subsection{Basic Elliptic Curve Cryptography}
We start with a few informal definitions \cite{3}. A more formal definition is provided by {4}, but is not necessary for the scope of this paper.

\begin{definition}{Elliptic Curve}\\
An elliptic curve, $H$, is defined as the set of points
$$H = \{(x, y) \in \mathbb{R}^{2} | y^{2} = x^{3} + ax + b, 4a^{3} + 27b^{2} \neq 0\} \cup \{0\}$$
Depending on the value of $a$ and $b$, elliptic curves may assume different shapes on a plane, but it can be easily verified that all elliptic curves that conform to this definition are symmetric about the x-axis. We implicitly defined our ideal point as ${0}$. This will be more clearly defined later.
\end{definition}

\begin{definition}{Abelian Group}
-define a group but only in the necessary context
\end{definition}

------define addition and multiplication operations on the points in an elliptic curve group (use the below lay explanation)\\

According to our definition of multiplication, it holds that given a number $a \in \mathbb{R}$, we can compute $a*H$ such that the product of this computation is a point on the elliptic curve $H$. For $b \in \mathbb{R}, b \neq a$, we can also calculate $(a+b)*H = a*H + b*H$. This implies that the addition and multiplication operations on an elliptic curve are associative and commutative. \\

In Elliptic Curve Cryptography (ECC), if we pick a large number $x$ as a private key, $x*H$ is considered the corresponding public key. According to the aforementioned properties of elliptic curve groups, this construction guarantees that even if one knows the value of the public key $x*H$, deducing $x$ is very difficult. This follows from the fact that while multiplication is trivial, division by curve points is very difficult (the hardness of this operation is well-studied and better known as the discrete logarithm problem \cite{5}). To see how easy multiplication is, notice how if $a$ amd $b$ are both private keys, a public key obtained from the addition of two private keys $(a+b)*H$ is identical to the addition of the public keys for each of the two private keys ($a*H + b*H$). With these properties and a new structure for storing transactions, we can implement the MimbleWimble protocol to guarantee strong privacy and confidentiality.

\subsection{MimbleWimble Transaction Structure}

MimbleWimble transaction validation relies on two important properties: \\
(1) The sum of outputs minus inputs always equal zero, proving that the transaction did not create new funds, without revealing the actual amounts.\\
(2) Ownership of transaction outputs is guaranteed by the possession of ECC private keys.\\

By applying the previously discussed properties of ECC, we can obscure the values in a transaction. Specifically, for any transaction input or output $x$ and an elliptic curve $H$, we can embed $x*H$ while still verifying that (1) holds. The verification of this property for every transaction guarantees completeness, but there are a finite number of usable values and an attacker could try a large amount of them in order to guess the value of a transaction. This attack vector is exacerbated by the fact that knowing an $x$ from a previous transaction (that the attacker may have been involved in) and the resulting $x*H$ reveals all the outputs with value $x$ across the entire distributed ledger. Therefore, MimbleWimble uses a second elliptic curve $G$ and a private key $r$ that's primary use is to obscure the transaction data.\\

With the restriction that $G$ is another generator point on the same curve group as $H$, an input or output value in a transaction $x$ can be expressed as\\
$$r*H + x*H$$
This is more formally known as a Pederson Commitment with blinding factor $r$ such that $r*G$ is the public key for $r$ on $G$. Neither $v$ nor $r$ can be deduced, thereby leveraging the fundamental properties of ECC.\\

Under this constructoion, while the private key introduced as a binding factor is primarily used to obscure the transaction's values, the private key can also be used to prove owership of a specific input or output. To understand how this might work, let's consider an example with Alice, Bob, and Eve. \\
//CREATE SOME SORT OF GRAPHIC FOR THIS EXCHANGE TO SHOW THIS WORKS....WITH PICTURES AND COLOR
If Alice sends Bob 3 coins and Bob chooses a very large number $p$ as his binding factor, then somewhere on the blockchain, the output $X = p*G + 3*H$ would only be spendable by Bob (because he is the only one who knows $p$). While $X$ is public and therefore visible to anyone looking at the blockchain, the value $3$ is known only by Alice and Bob and only Bob knows $p$. This construction implies that to transfer those coins again, $p$ must be known by the sender. Specifically, there is not way to build such a transaction and balance it without knowing the private key $p$. Therefore, if Bob wanted to send the 3 coins to Eve, Eve must know both the value sent (3 coins in this case) as well as the private key $p$.
//DIAGRAM OF AN INPUT TO AN OUTPUT (WHERE THE INPUT SPENDS YOUR OUTPUT..THINK ABOUT THIS NOTATION AND ALSO USE IT TO ILLUSTRATE WHAT HAPPENED WHEN ALICE SENT TO BOB INITIALLY)//
...CONTINUE THIS UNTIL DONE

Based on this ECC construction, the protocol only needs to veiryf that 

-------------discuss how this idea is derived from Greg MaxWell's Confidential transactions, which itself is derived from an Adam Back proposal for homorphic values applied to Bitcoin...OR JUST CITE IT?

WE WOULD NEED TO FIGURE OUT HOW TO CODE ALL THIS ELLIPTIC CURVE STUFF INTO THE SMART CONTRACT CODE. THE CODE WOULD BASICALLY PROVIDE FUNCTIONS TO INTERACT WITH A CHAIN THAT CONSIST OF A GOSSIP PROTOCOL WHERE A BLOCK LEADER SENDS A BLOCK HASH TO THE MAIN CHAIN EVERY SO MANY TRANSACTIONS AND THEN ANY OF THE PARTICIPANTS CAN PROVIDE A PROOF THAT THE HASH SUBMITTED TO THE MAIN CHAIN WAS FALSE AND THIS IS USED TO REVERSE THE PREVIOUS INVALID STATE TRANSITION (AND CORRECT IT USING THE PROOF). THEN, THINGS WOULD NEED TO PROCEED AS USUAL.
.....HARD PART ABOUT CODING THIS ARE (1) OWNERSHIP AND PREVENTION OF REENTRANCY (2) THE ECC MIMBLEWIMBLE PROTOCOL IN SOLIDITY'S CONTRACT LANGUAGE AND (3) EXTENSIVE TIMELOCKS FOR THE PROOFS (SIMILAR TO CASPER, THE TIMELOCKS LOCK UP LIQUIDITY IN ORDER TO PROVIDE GUARANTEES OF CONSISTENCY TO DILIGENT PARTICIPANTS IN THE PLASMA CHAIN (WHO EITHER PERFORM THE COMPUTATION OR VERIFY IT USING BULLETPROOFS...))

\subsection{}



\subsection{Implementing BulletProofs to Increase the Practicality of Range Proofs for MimbleWimble}
-benedict buenz's paper



\begin{thebibliography}{9}
\bibitem{1}
	Joseph Poon, Vitalik Buterin. 2017. Plasma: Scalable Autonomous Smart Contracts. 
	https://plasma.io/plasma.pdf
\bibitem{2}
	Grin. https://github.com/mimblewimble/grin/blob/master/doc/intro.md
\bibitem{3}
	Corbellini_assumptions. http://andrea.corbellini.name/2015/05/17/elliptic-curve-cryptography-a-gentle-introduction/.
\bibitem{4}
	Elliptic_curve_formal_defintion. http://mathworld.wolfram.com/EllipticCurve.html

\end{thebibliography}
\end{document}






